%%%%%%%%%%
% Ravi S. Jonnal - http://pointsofsail.org - rjonnal@indiana.edu
% Preparing your CV using LaTeX and BibTeX. 
%
% URL: http://pointsofsail.org/wikka.php?wakka=LatexCV
% Some rights reserved: http://creativecommons.org/licenses/by-sa/3.0/

% Much of this template was borrowed from or inspired by Dario
% Taraborelli's excellent article "Typesetting your academic CV in
% LaTeX" (http://nitens.org/taraborelli/cvtex) and the associated
% template, hence the by-sa license above. Where code was lifted
% directly, an attribution is included. Copied code is indicated with a
% '% *DT' following the line.
%
% Some key differences are noteworthy, and these differences represent
% the utility of this template (if it has any):
%
% 1. It is written in plain LaTeX rather than XeTeX. While the latter
% is included in most LaTeX distributions (texlive, miktex, and
% others), I felt that a plain LaTeX template may be useful for some
% people using legacy systems. There are many good reasons for using
% XeTeX, chiefly the availability of expert fonts instead of LaTeX's
% native fonts, which are problematic. (See
% http://cs.gmu.edu/~sean/stuff/DoingLatexRight/ for a discussion of
% some of these problems.)
%
% 2. It generates publications, proceedings, and presentations from a
% BibTeX database, using the very nice bibentry package. bibentry is
% part of the natbib package, which must be installed.
%
% Note: it may be necessary to insert \pagebreak[4] at places where
% page breaks are desired. Because of an oddity with \marginpar, pages
% are occasionally broken between the marginal year and the item
% associated with that year. Placing \pagebreak[4] strategically can
% avoid this problem. If you have a more elegant solution, please do
% let me know.
%
%%%%%%%%%%

\documentclass[10pt, letterpaper]{article}


\usepackage{bibentry} % must have natbib package installed

\usepackage{multicol} % part of most LaTeX distributions

\usepackage[T1]{fontenc} % part of most LaTeX distributions
% The choice of font below will be constrained by available fonts on
% your system.  

%\usepackage{palatino}
%\usepackage{tgschola} 
%\usepackage{venturisold}
\usepackage{bookman}
%\usepackage{georgia}

%% Layout margins, marginal notes
\usepackage{geometry} % part of most LaTeX distributions

%============= kp
%\usepackage{hyperref}
\usepackage{verbatim} % For multiline comments: http://www.devdaily.com/blog/post/latex/multi-line-comments-in-latex-begin-123-comment-125-verbatim
%============= kp

\geometry{letterpaper, marginparsep=7pt, marginparwidth=.6in, textwidth=5.75in, textheight=8in}

%% Do not indent
\setlength\parindent{0in}

%% MARGIN YEARS
\usepackage{marginnote} % *DT
\newcommand{\years}[1]{\marginnote{\small #1}} % *DT

%% PUBLICATIONS, ETC. - takes two arguments; the first is the year of
%%%% publication and the second is the bibtex label
\newcommand{\publication}[2]{ \years{#1} \bibentry{#2}. \par\vspace{0.8mm}  }
\newcommand{\publicationNyr}[1]{ \bibentry{#1}. \par\vspace{0.8mm}  }

\newcommand{\cd}{$\cdot${ }}

\renewcommand*{\raggedleftmarginnote}{} % *DT
%\setlength{\marginparsep}{7pt} % *DT
\reversemarginpar % *DT


% HEADINGS
\usepackage{sectsty}

\usepackage[normalem]{ulem}

\sectionfont{\mdseries\upshape\Large}
\subsectionfont{\mdseries\scshape\normalsize}
\subsubsectionfont{\mdseries\upshape\large}


%\input{/home/kpadhikari/Dropbox/TestFiles/Texmaker/kpAddedPackages.tex} 
%%
% kpa: 8/14/13:
% This file I copied from my Dropbox/TestFiles/Textgreek, where I generated it by running 'latex textgreek.ins'
%    which I downloaded from http://www.ctan.org/tex-archive/macros/latex/contrib/textgreek
%     (which in turn was reached from http://tex.stackexchange.com/questions/9043/should-greek-letters-inserted-in-text-using-math-mode-mostly-always-be-italic)
% I got to that page from google, while trying to get small 'theta' in the text mode (like that given by $\theta$ 
%    in the math mode but without italicization.
%

%%
%% This is file `textgreek.sty',
%% generated with the docstrip utility.
%%
%% The original source files were:
%%
%% textgreek.dtx  (with options: `package')
%% 
%% This is a generated file.
%% 
%% Copyright 2010,2011 Leonard Michlmayr
%% 
%% This work may be distributed and/or modified under the
%% conditions of the LaTeX Project Public License, either version 1.3
%% of this license or (at your option) any later version.
%% The latest version of this license is in
%% 
%%   http://www.latex-project.org/lppl.txt
%% 
%% and version 1.3 or later is part of all distributions of LaTeX
%% version 2005/12/01 or later.
%% 
%% This work has the LPPL maintenance status `author-maintained'.
%% 
%% The Current Maintainer of this work is Leonard Michlmayr
%% <leonard.michlmayr at gmail.com>.
%% 
%% This work consists of the files textgreek.dtx and textgreek.ins
%% and the derived files textgreek.sty and textgreek.pdf
%% 
\NeedsTeXFormat{LaTeX2e}[2009/09/24]
\ProvidesPackage{textgreek}
   [2011/10/09 v0.7 Greek symbols in text]
\InputIfFileExists{lgrenc.def}{%
  \PackageInfo{textgreek}{Loading the definitions for the Greek font%
    encoding.}}{%
  \PackageError{textgreek}{Cannot find the file lgrenc.def}{%
    lgrenc.def is a file that contains the definitions for the Greek
    font encoding LGR. Maybe it comes with the babel package.}}
\DeclareOption{cbgreek}{%
\renewcommand*{\textgreekfontmap}{%
  {eur/*/*}{U/eur/*/*}
  {phv/*/*}{U/psy/*/*}}}%
\DeclareOption{euler}{%
\renewcommand*{\textgreekfontmap}{%
  {phv/*/*}{U/psy/*/*}
  {*/bx/n}{U/eur/b/n}
  {*/b/n}{U/eur/b/n}
  {*/*/n}{U/eur/m/n}
  {*/*/it}{OML/*/*/*}}}%
\DeclareOption{artemisia}{%
\renewcommand*{\textgreekfontmap}{%
  {eur/*/*}{U/eur/*/*}
  {phv/*/*}{U/psy/*/*}
  {*/b/n}{LGR/artemisia/b/n}
  {*/bx/n}{LGR/artemisia/bx/n}
  {*/*/n}{LGR/artemisia/m/n}
  {*/b/it}{LGR/artemisia/b/it}
  {*/bx/it}{LGR/artemisia/bx/it}
  {*/*/it}{LGR/artemisia/m/it}
  {*/b/sl}{LGR/artemisia/b/sl}
  {*/bx/sl}{LGR/artemisia/bx/sl}
  {*/*/sl}{LGR/artemisia/m/sl}
  {*/*/sc}{LGR/artemisia/m/sc}
  {*/*/sco}{LGR/artemisia/m/sco}}}%
\newcommand*{\textgreekfontmap}{}%
\ExecuteOptions{cbgreek}
\ProcessOptions\relax%
\PackageInfo{textgreek}{Loaded fontmap: \textgreekfontmap.}
\def\textgreek@setfont#1/#2/#3/#4\relax{%
\textgreek@ematch{#1}{*}{}{\fontencoding{#1}}%
\textgreek@ematch{#2}{*}{}{\fontfamily{#2}}%
\textgreek@ematch{#3}{*}{}{\fontseries{#3}}%
\textgreek@ematch{#4}{*}{}{\fontshape{#4}}}%
\def\textgreek@eof{}%
\def\textgreek@return#1#2\textgreek@eof{%
\fi#1}%
\def\textgreek@ematch#1#2#3#4{%
  \begingroup%
  \edef\tempa{#1}\edef\tempb{#2}\def\tempc{*}%
  \def\return##1##2\endgroup{\fi\endgroup##1}%
  \ifx\tempa\tempb\return{#3}\fi%
  \ifx\tempa\tempc\return{#3}\fi%
  \iftrue\return{#4}\fi%
  \endgroup}%
\def\textgreek@matchfont#1/#2/#3\relax#4#5{%
  \textgreek@ematch{#1}{\f@family}{%
    \textgreek@ematch{#2}{\f@series}{%
      \textgreek@ematch{#3}{\f@shape}{#4}{#5}}%
    {#5}}%
  {#5}%
}%
\def\textgreek@findfont@#1#2#3\textgreek@eof{%
  \textgreek@matchfont#1\relax%
  {\textgreek@setfont#2\relax}%
  {\textgreek@findfont#3\textgreek@eof}}%
\def\textgreek@findfont#1\textgreek@eof{%
  \begingroup%
  \def\temp{#1}%
  \def\return##1##2\endgroup{\fi\endgroup##1}%
  \ifx\temp\textgreek@eof\else%
  \return{\textgreek@findfont@#1\textgreek@eof}%
  \fi\endgroup}%
\newcommand*{\textgreekfont}{%
  \fontencoding{LGR}%
  \edef\textgreek@fontmap{\textgreekfontmap}%
  \expandafter\textgreek@findfont\textgreek@fontmap\textgreek@eof%
  \selectfont%
}%
\DeclareRobustCommand*{\TextGreek}[1]{%
\begingroup%
\textgreekfont%
\edef\tempa{\f@family}%
\let\tempd\f@encoding%
\def\tempb{eur}\def\tempc{psy}%
\def\tempe{OML}%
\ifx\tempd\tempe\textgreek@return{\lgrtoeuler#1}\fi%
\ifx\tempa\tempb\textgreek@return{\lgrtoeuler#1}\fi%
\ifx\tempa\tempc\textgreek@return{\lgrtosymbol#1}\fi%
#1%
\textgreek@eof%
\endgroup}%
\DeclareRobustCommand*{\TextGreek@Select}[3]{%
\begingroup%
\textgreekfont%
\edef\tempa{\f@family}%
\let\tempd\f@encoding%
\def\tempb{eur}\def\tempc{psy}%
\def\tempe{OML}%
\ifx\tempd\tempe\textgreek@return{#2}\fi%
\ifx\tempa\tempb\textgreek@return{#2}\fi%
\ifx\tempa\tempc\textgreek@return{#3}\fi%
#1%
\textgreek@eof%
\endgroup}%
\def\DeclareTextGreekSymbol#1#2{%
  \@ifnextchar[%
  {\DeclareTextGreekSymbol@{#1}{#2}}%
  {\@DeclareTextGreekSymbol{#1}{#2}%
    {\fontencoding{LGR}\fontfamily{cmr}\selectfont#2}{#2}}%
}%
\def\@DeclareTextGreekSymbol#1#2#3#4{%
  \expandafter\DeclareTextCommandDefault\csname text#1\endcsname%
  {\TextGreek@Select{#2}{#3}{#4}}%
}%
\def\DeclareTextGreekSymbol@#1#2[#3]{%
  \ifx\textgreek@eof#3\textgreek@return{%
    \DeclareTextGreekSymbol@@{#1}{#2}%
    {\fontencoding{LGR}\fontfamily{cmr}\selectfont#2}}%
  \else\textgreek@return{%
    \DeclareTextGreekSymbol@@{#1}{#2}{#3}}\fi%
  \textgreek@eof}%
\def\DeclareTextGreekSymbol@@#1#2#3{%
  \@ifnextchar[%
  {\DeclareTextGreekSymbol@@@{#1}{#2}{#3}}%
  {\@DeclareTextGreekSymbol{#1}{#2}{#3}{#2}}%
}%
\def\DeclareTextGreekSymbol@@@#1#2#3[#4]{%
  \ifx\textgreek@eof#4\textgreek@return{%
    \@DeclareTextGreekSymbol{#1}{#2}{#3}{#2}}%
  \else\textgreek@return{%
    \@DeclareTextGreekSymbol{#1}{#2}{#3}{#4}}\fi%
  \textgreek@eof}%
\DeclareTextGreekSymbol{alpha}{a}[\char11]
\DeclareTextGreekSymbol{beta}{b}[\char12]
\DeclareTextGreekSymbol{gamma}{g}[\char13]
\DeclareTextGreekSymbol{delta}{d}[\char14]
\DeclareTextGreekSymbol{epsilon}{e}[\char34]
\DeclareTextGreekSymbol{zeta}{z}[\char16]
\DeclareTextGreekSymbol{eta}{h}[\char17]
\DeclareTextGreekSymbol{theta}{j}[\char18][q]
\DeclareTextGreekSymbol{iota}{i}[\char19]
\DeclareTextGreekSymbol{kappa}{k}[\char20]
\DeclareTextGreekSymbol{lambda}{l}[\char21]
\expandafter\ifx\csname?\string\textmu\endcsname\relax%
\DeclareTextGreekSymbol{mu}{m}[\char22]
\fi
\DeclareTextGreekSymbol{mugreek}{m}[\char22]
\DeclareTextGreekSymbol{nu}{n}[\char23]
\DeclareTextGreekSymbol{xi}{x}[\char24]
\DeclareTextGreekSymbol{omikron}{o}
\DeclareTextGreekSymbol{pi}{p}[\char25]
\DeclareTextGreekSymbol{rho}{r}[\char26]
\DeclareTextGreekSymbol{sigma}{s\noboundary}[\char27][s]
\DeclareTextGreekSymbol{varsigma}{c}[][V]
\DeclareTextGreekSymbol{tau}{t}[\char28]
\DeclareTextGreekSymbol{upsilon}{u}[\char29]
\DeclareTextGreekSymbol{phi}{f}[\char39][j]
\DeclareTextGreekSymbol{chi}{q}[\char31][c]
\DeclareTextGreekSymbol{psi}{y}[\char32]
\DeclareTextGreekSymbol{omega}{w}[\char33]
\DeclareTextGreekSymbol{Alpha}{A}
\DeclareTextGreekSymbol{Beta}{B}
\DeclareTextGreekSymbol{Gamma}{G}[\char0]
\DeclareTextGreekSymbol{Delta}{D}[\char1]
\DeclareTextGreekSymbol{Epsilon}{E}
\DeclareTextGreekSymbol{Zeta}{Z}
\DeclareTextGreekSymbol{Eta}{H}
\DeclareTextGreekSymbol{Theta}{J}[\char2][Q]
\DeclareTextGreekSymbol{Iota}{I}
\DeclareTextGreekSymbol{Kappa}{K}
\DeclareTextGreekSymbol{Lambda}{L}[\char3]
\DeclareTextGreekSymbol{Mu}{M}
\DeclareTextGreekSymbol{Nu}{N}
\DeclareTextGreekSymbol{Xi}{X}[\char4]
\DeclareTextGreekSymbol{Omikron}{O}
\DeclareTextGreekSymbol{Pi}{P}[\char5]
\DeclareTextGreekSymbol{Rho}{R}
\DeclareTextGreekSymbol{Sigma}{S}[\char6]
\DeclareTextGreekSymbol{Tau}{T}
\DeclareTextGreekSymbol{Upsilon}{U}[\char7]
\DeclareTextGreekSymbol{Phi}{F}[\char8]
\DeclareTextGreekSymbol{Chi}{Q}[][C]
\DeclareTextGreekSymbol{Psi}{Y}[\char9]
\DeclareTextGreekSymbol{Omega}{W}[\char10]
\DeclareTextCommand{\straightphi}{PU}%
  {\83\325} % U+03D5 GREEK PHI SYMBOL
\DeclareTextCommandDefault{\straightphi}{%
\begingroup\textgreekfont%
\edef\tempa{\f@family}%
\edef\tempb{\f@shape}%
\def\tempc{eur}\def\tempd{psy}%
\def\tempe{it}%
\ifx\tempa\tempc\textgreek@return{\char30}\fi%
\ifx\tempa\tempd\textgreek@return{f}\fi%
\ifx\tempb\tempe\textgreek@return{%
  \fontencoding{OML}\selectfont\char30}\fi%
\textgreek@ematch{\f@series}{bx}{\fontseries{b}}{}%
\fontencoding{U}\fontfamily{eur}\selectfont\char30%
\textgreek@eof%
\endgroup}%
\DeclareTextCommand{\scripttheta}{PU}%
  {\83\321}% U+03D1 GREEK THETA SYMBOL
\DeclareTextCommandDefault{\scripttheta}{%
  \TextGreek@Select{j}{\char35}{J}}%
\DeclareTextCommand{\straighttheta}{PU}%
  {\83\270} % U+03B8 GREEK THETA SYMBOL
\DeclareTextCommandDefault{\straighttheta}{%
\begingroup\textgreekfont%
\edef\tempa{\f@family}%
\edef\tempb{\f@shape}%
\def\tempc{eur}\def\tempd{psy}%
\def\tempe{it}%
\ifx\tempa\tempc\textgreek@return{\char18}\fi%
\ifx\tempa\tempd\textgreek@return{q}\fi%
\ifx\tempb\tempe\textgreek@return{%
  \fontencoding{OML}\selectfont\char18}\fi%
\textgreek@ematch{\f@series}{bx}{\fontseries{b}}{}%
\fontencoding{U}\fontfamily{eur}\selectfont\char18%
\textgreek@eof%
\endgroup}%
%% U+03F5 GREEK LUNATE EPSILON SYMBOL
\DeclareTextCommand{\straightepsilon}{PU}{\83\365}%
\DeclareTextCommandDefault{\straightepsilon}{%
\begingroup\textgreekfont%
\edef\tempa{\f@family}%
\edef\tempb{\f@shape}%
\def\tempc{eur}\def\tempd{psy}%
\def\tempe{it}%
\ifx\tempa\tempc\textgreek@return{\char15}\fi%
\ifx\tempa\tempd\textgreek@return{%
  \fontfamily{eur}\fontseries{b}\selectfont\char15}\fi%
\ifx\tempb\tempe\textgreek@return{%
  \fontencoding{OML}\selectfont\char15}\fi%
\textgreek@ematch{\f@series}{bx}{\fontseries{b}}{}%
\fontencoding{U}\fontfamily{eur}\selectfont\char15%
\textgreek@eof%
\endgroup}%
\endinput
%%
%% End of file `textgreek.sty'.
 

% DOCUMENT
\begin{document}

%\nobibliography{template_bibliography.bib}
\nobibliography{bibsOfPapersWithMyName.bib}

{\LARGE Krishna P. Adhikari, \emph{Ph.D. }} \\[.5cm]
%%%%%Department of Physics, Old Dominion University\\[0.2cm]
%%%%4600 Elkhorn Ave. Rm 306 \\ Norfolk, VA, 23529 \\
7 Dorsey Rd Apt D \\
Newport News VA, 23606 \\
%4707 Killam Ave Apt B4 \\ Norfolk, VA, 23508 \\
%Phone: 757 839 8594\\ %email: kadhi001@odu.edu
Phone (Cell): (757) 839 8594 \\
%Fax: (757) 683-3038 \\
Email: \texttt{kadhi001@odu.edu, adhikari@jlab.org} \\


% =========== This is an addition from my earlier version of CV to the existing template 
% \section*{Personal}
% \begin{itemize}
%\item Born on July 30, 1976 (Chitwan, Nepal).
% \item Nepalese Citizen.
% \end{itemize}
% ===========  (perhaps unncessary because of the following section on Education which shows Nepal twice) 


%\section*{Areas of specialization}
%\noindent
%Pranks \cd Foul jokes \cd Ennui



\section*{Education}
\noindent
    \textbf{Ph.D.}, Physics, Old Dominion University, Norfolk, VA, USA.  \hfill Dec 2013\\
    %\begin{Verbatim}
       \hspace*{5 mm} \textbf{Dissertation:} Measurement of the spin structure function $g^D_{1}$ \\
       \hspace*{10 mm} of the deuteron and its moments at low $Q^2$. \\
       \hspace*{5 mm} \textbf{Supervisor:} Dr. Sebastian E. Kuhn (\texttt{kuhn@jlab.org}) \\
    %\end{Verbatim}
    \textbf{M.S.}, Physics, Old Dominion University, Norfolk, VA, USA. \hfill 2006 \\ 
    \textbf{M.Sc.}, Physics, Tribhuvan University, Kathmandu, Nepal. \hfill 2000 \\ 
    \textbf{B.Sc.}, Physics, Mathematics, Chemistry, Tribhuvan University, Nepal. \hfill 1997 %\\ %\years{1997}


\section*{Work Experience}
\noindent
		{\bf Research Experience:} \\
		%And, just recently, I have joined the JLab’s HD-ice group for a few months as a “Casual” TD-I to gain some skills and experience in cryogenics, the new polarized target technology and hardware in general. 
		Postdoctoral Research Associate (Dipangkar Dutta, Supervisor), \\
		 \hspace*{3 mm} Mississippi State University \hfill Since Sep 2014 \\
	    Temporary technical worker in JLab's HDice group \hfill Feb-Aug 2014\\
		Graduate Research Assistant (S. Kuhn, Supervisor), \\
		\hspace*{3 mm} Old Dominion University \hfill 2007 - 2013 \\
		Won prestigious fellowship for DOE/INFN Summer Exchange Program, \\
		\hspace*{3 mm} INFN, Genoa, Italy \hfill 2009 \\
		Participated in studying Insolation and Photovoltaic system, TU, Nepal. \hfill 2003 \\
		Assisted a senior student in operating Brewer Spectrometer, TU, Nepal. \hfill 2000 \\

		{\bf Teaching Experience:} \\
		Teaching Assistant, Physics, Old Dominion University \hfill 2004 - 2007 \\
		Part-time lecturer of undergraduate level Physics\\
		\hspace*{3 mm} Amrit Science College, Thamel, Kathmandu, Nepal. \hfill 2002 - 2004 \\
		\hspace*{3 mm} Asian College of Higher Studies, Kathmandu, Nepal. \hfill 2002 - 2004 \\
		\hspace*{3 mm} Patan Multiple Campus, Lalitpur, Nepal. \hfill 2002 - 2004 \\
		\hspace*{3 mm} BernHardt College, Kalimati, Kathmandu, Nepal. \hfill 2001 - 2004 \\
		Science and Mathematics teacher, \\
		\hspace*{3 mm} Naya Kiran Secondary School, Chitwan, Nepal.               \hfill 1997 - 1998





\section*{Computer/Programming Skills/Training}
\noindent
%\begin{itemize}
%\item Windows, Linux, C/C++, Fortran, Java/Eclipse, Perl, Make, ROOT, GEANT, HTML, JavaScript, LaTex
%\end{itemize}
\begin{itemize}
%kp: \item ROOT - an object-oriented program and library developed by CERN, a framework designed for particle physics data analysis.
\item Working knowledge of the following programming languages and tools
  \begin{itemize} 
  \item \textbf{Operating systems}: Microsoft Windows, Linux/Unix 
  \item \textbf{Office Tools}: Microsoft Office, OpenOffice.org
  \item \textbf{Programming Languages}: C/C++, Fortran, Visual Basic (VB.NET, Visual Studio), Java (Eclipse), Shell Scripts (Perl, bash, tcsh, Makefile, CMake)
  \item \textbf{Graphic Editors}: SketchUp, Inkscape, JaxoDraw
  \item \textbf{Text Formatting (Markup) languages}: LaTex
  \item \textbf{Web Languages}: HTML, XHTML, JavaScript %, CSS, PHP, ASP
  \item \textbf{Software packages}:
    \begin{itemize}
    \item \textbf{ROOT} - an object-oriented program and library developed by CERN, a framework designed for particle physics data analysis.
    \item \textbf{GEANT4} - a C++ based toolkit for the simulation of the passage of particles through matter
    \item \textbf{GSIM} - a GEANT3 based package for the Monte Carlo simulation of the CLAS detector in Jlab.	
    \end{itemize}
\end{itemize}
%kp: \item \href{http://jblevins.org/research/pdest}{PDESolutionTester},  A Mathematica program for the symbolic verification of exact solutions
%kp:   of nonlinear partial differential equations,  with Jeff Heath and Willy Hereman (2002).
\end{itemize}




\section*{Schools \& Workshops}
\noindent
\begin{itemize} 
\item GEANT4 Tutorial, Jefferson Lab, Newport News, VA,  \hfill 2012
\item National Nuclear Physics Summer School (NNPSS), Chapel Hill, NC,  \hfill 2011
\item CLAS12 Software Workshop, U. of Richmond, \hfill May 2010 %http://www.jlab.org/conferences/CLAS12Software/
\item DOE/INFN Summer Exchange Program, INFN, Genoa, Italy,  \hfill 2009
\item Hampton University Graduate Summer School (HUGS), \\
\hspace*{3 mm} Jefferson Lab, Newport News, VA,  \hfill 2009
\item International Workshop on Positrons at Jefferson Lab, \\
\hspace*{3 mm} Newport News, VA,  \hfill 2009 %http://www.jlab.org/conferences/JPOS09/
\item Spin Structure at Long Distance (Workshop), Jefferson Lab, \\
\hspace*{3 mm} Newport News, VA,  \hfill 2009
\end{itemize}



\section*{Trainings/Certifications}
\noindent
\begin{itemize}
\item Radiation and Lead Worker trainings, JLab
\item Oxygen Deficiency Hazards, JLab, \hfill Since 2008
\item Fire and Ladder Safety trainings, JLab, \hfill June 2014
\item Pressure Systems Awareness training (SAF130A), JLab, \hfill June 2014
\item Responsible Conduct of Research: Physical Science Research \hfill 2013
\item Leadership Training (Office of Student Activities and \\
\hspace*{3 mm} Leadership, ODU), \hfill Nov 2007
\end{itemize}




\section*{Honors, Awards, \& Fellowships}
\noindent
%\years{2011} Award 1, Award agency 1\\
%\years{2005} Outstanding Scorer in Ph.D. Qualifying Exam, Dept. Of Physics, Old Dominion University \\
\begin{itemize}
\item Award for highest score (Ph. D. Qualifying Exam), Department of Physics, Old Dominion University \hfill 2005
%\item Hampton University Graduate Studies Fellwoship for HUGS Summer School, Newport News, VA \hfill June  2009
\item INFN Scholarship for the DOE/INFN Summer Exchange Program, INFN, Genova, Italy \hfill 2009 
\end{itemize}




\section*{Leadership, Volunteering and Community Service}
\noindent
  {\bf Volunteering / Community Service: } \\
  Showcased several ODU Physics Demos during multiple Norfolk Children \\
  \hspace*{3 mm} Festivals, and Open Houses at ODU and Jefferson Lab. \\
  Monarch Mentor to new International Students, Office of Inter-cultural \\
  \hspace*{3 mm} Relations, Old Dominion University, 2006.\\
  {\bf Leadership Activities: } \\
  Monarch Mentor Training (2006) and Leadership Training (2007) \\
  \hspace*{3 mm} (from the Office of Inter-cultural Relations, Old Dominion University).\\
  Helped found Nepali Student Community (NeSCo) at ODU \\
  \hspace*{3 mm} (served as the founding President) 



\section*{Affiliations}
\noindent
%\item ODU Student Organization Affiliations
\begin{itemize}
  \item American Physical Society, Current Member 
  \item Nepal Physical Society, Life Member %(Since 2004)
  \item CLAS Collaboration, Term Member %(Since 2007)
  \item Past Member - Physics Graduate Student Association (PGSA)
  \item Founding President - Nepali Student Community (NeSCo)
  \item Past Member - Global Student Friendship (GSF)
%\item Life member, Nepal Physical Society (http://www.nps.org.np/index\_Page354.htm).
%\item Member, American Physical Society, 2005 - present.
%\item Member, Nuclear Physics of the American Physical Society, 2005 - present.
%\item Term Member, CLAS Collaboration, 2007 - present
%\item Member - Physics Graduate Student Association (Old Dominion University), 2007 - present.
\end{itemize}

%\section*{Patents}
%\noindent
%\years{2008} U.S. Patent 1,234,567,890,123. ``Method and apparatus something incredibly ingenious but totally unmarketable''.




%\section*{Publications \& talks}
%\section*{Publications \& talks}
%\subsection*{Conference presentations}
\section*{Conference presentations}
\noindent
%\begin{comment}
\begin{itemize}
\item K. Adhikari.  Deuteron Spin Structure function $g_1$ at low $Q^2$. APS October Meeting, Newport News, VA.  \hfill  2013
\item K. Adhikari.  Deuteron Spin Structure function $g_1$ at low $Q^2$. APS October Meeting,   Newport Beach, CA. \hfill   2012
\item Mapping Generalized GDH Integral at Low $Q^2$, Poster presentation at National Nuclear Physics Summer School (NNPSS), Chapel Hill, NC. \hfill  2011 %\href{http://www.int.washington.edu/NNPSS/schools.html}{National Nuclear Physics Summer School} (NNPSS), 2011 
\item EG4 Experiment, Poster presentation at Gordon Research Conference on Photonuclear Reactions, Tilton School, NH \hfill   2010 %Dates: 08/01/2010 - 08/06/2010 Location: Tilton School in Tilton NH United States 
%\item EG4-Analysis Update presentations to several CLAS Collaboration meetings 

%\item EG4-update (Oct 12, 2012) for DPWG meeting by K. P. Adhikari.
%\item EG4-Status-Update presentation, \href{http://www.jlab.org/Hall-B/claschair/Clas\_Chair\_page.htm}{CLAS Collaboration} - Deep Processes Working Group Meeting, Oct 12, 2012
\item EG4-Status-Update presentation, CLAS Collaboration - Deep Processes Working Group Meeting, Oct 12 \hfill    2012
\item EG4-Status-Update presentation, CLAS Collaboration - Deep Processes Working Group Meeting \hfill   Mar 19, 2010
\item EG4-Status-Update presentation, CLAS Collaboration - Deep Processes Working Group Meeting \hfill   Feb 23, 2008
\item EG4-Status-Update presentation, CLAS Collaboration - Deep Processes Working Group Meeting \hfill    Nov 02, 2007 

%\item \textbf{Study of Insolation on Kirtipur and the Performance of Photovoltaic Systems}, Seminar, 4th National Conference on Science and Technology organized by Royal Nepal Academy of Science and Technology (RONAST), March 2004.
%\item \textbf{An attempt toward the half-life of ozone}, Seminar, 20th annual meeting of Nepal Physical Society, 2003.
%\item \textbf{Superconductivity}, Physics Department Graduate Student Seminar, ODU, Nov 10, 2004.
%\item \textbf{EPR Paradox and Bell's Theorem}, Physics Department Graduate Student Seminar, ODU, Nov 9, 2005.
%\item \textbf{EG4 Update}, Jlab CLAS Collaboration (Deep Processes Working Group) meeting, Nov 2, 2007.
%\item \textbf{EG4 Update}, Jlab CLAS Collaboration (Deep Processes Working Group) meeting, Feb 23, 2008. 
%\item \textbf{EG4 experiment}, Nuclear Physics Student Seminar, ODU, July 30, 2008.
%\item \textbf{EG4 Update}, Nuclear Physics Student Seminar, ODU, June, 2009.
%\item \textbf{EG4 - a low $Q^2$ CLAS- GDH- measurement}, HUGS Student Seminar, Jlab, June 17, 2009.
%\item \textbf{EG4 Update}, Jlab CLAS Collaboration (Deep Processes Working Group) meeting, March 19, 2010.
%\item \textbf{EG4 - Mapping Generalized GDH Sum Rule at Low $Q^2$}, Poster Presentation, Gordon Research Conference on Photonuclear Reactions, Tilton School, Tilton, NH, Aug 1-6, 2010.
%\item \textbf{EG4 - Mapping Generalized GDH Sum Rule at Low $Q^2$}, Poster Presentation, The 2011 National Nuclear Physics Summer School, University of North Carolina, Chapel Hill, NC, June 20 - July 1, 2011. %http://www.tunl.duke.edu/nnpss/

\end{itemize}
%\end{comment}








%\subsection*{Journal articles, first author}
%\noindent




%\subsection*{Journal articles, co-author}
%\subsection*{Refereed Publications}
\section*{Refereed Publications}
\noindent
\textbf{Co-author on more than 40 CLAS publication papers} 

%\subsubsection*{2013 Publications}
\subsection*{2013 Publications}
\noindent
%\publication{2009}{key_for_publication_3}

%\publication{2011}{key_for_publication_1}
%\publication{2010}{key_for_publication_2}
\publicationNyr{Dugger:2013crn}
\publicationNyr{Adhikari:2013ija}
\publicationNyr{Moteabbed:2013isu}
\publicationNyr{Moriya:2013hwg}
\publicationNyr{Pomerantz:2013ysu}
\publicationNyr{Tang:2013gsa}
\publicationNyr{Nepali:2013bp}
\publicationNyr{Moriya:2013eb}
%%%%%%%  \publication{2012}{Hen:2012yva}
%%%%%%%  \publication{2012}{Khetarpal:2012vs}
%%%%%%%  \publication{2012}{Bedlinskiy:2012be}
%%%%%%%  \publication{2012}{Park:2012rn}
%%%%%%%  \publication{2012}{Mokeev:2012vsa}
%%%%%%%  \publication{2012}{Baghdasaryan:2012yn}
%%%%%%%  \publication{2012}{ElFassi:2012nr}
%%%%%%%  \publication{2012}{Park:2012yf}
%%%%%%%  \publication{2011}{Keller:2011aw}
%%%%%%%  \publication{2011}{Daniel:2011nq}
%%%%%%%  \publication{2011}{Baillie:2011za}
%%%%%%%  \publication{2011}{CLAS:2011aa}
%%%%%%%  \publication{2011}{Keller:2011nt}
%%%%%%%  \publication{2010}{Qian:2010rr}
%%%%%%%  \publication{2010}{Nasseripour:2010ne}
%%%%%%%  \publication{2010}{Baghdasaryan:2010nv}
%%%%%%%  \publication{2010}{Wood:2010ei}
%%%%%%%  \publication{2010}{Dey:2010hh}
%%%%%%%  \publication{2010}{Avakian:2010ae}
%%%%%%%  \publication{2010}{Osipenko:2010sb}
%%%%%%%  \publication{2009}{AnefalosPereira:2009zw}
%%%%%%%  \publication{2009}{McCracken:2009ra}
%%%%%%%  \publication{2009}{Aznauryan:2009mx}
%%%%%%%  \publication{2009}{Williams:2009yj}
%%%%%%%  \publication{2009}{Williams:2009aa}
%%%%%%%  \publication{2009}{Williams:2009ab}
%%%%%%%  \publication{2009}{Nasseripour:2009in}
%%%%%%%  \publication{2009}{Qian:2009ab}
%%%%%%%  \publication{2009}{Battaglieri:2009aa}
%%%%%%%  \publication{2009}{Carman:2009fi}
%%%%%%%  \publication{2008}{Lachniet:2008qf}
%%%%%%%  \publication{2008}{Battaglieri:2008ps}
%%%%%%%  \publication{2008}{Morrow:2008ek}
\bibliographystyle{plain}


\subsection*{Additional Publications}
\noindent
%\textbf{For more, please look at} \\
 \url{http://inspirehep.net/author/profile/K.P.Adhikari.1}

%\subsubsection{Older CLAS publications}
%\begin{itemize}
%\item %Co-author on more than 40 CLAS publication papers: \\
%\url{http://inspirehep.net/author/profile/K.P.Adhikari.1}%{http://inspirehep.net/author/profile/K.P.Adhikari.1}
%\end{itemize}





\section*{References}
\textbf{Dr. Sebastian E. Kuhn}, Professor, Old Dominion University, 4600 Elkhorn Ave, Room 306, Norfolk, VA 23529 \\
Email: \texttt{skuhn@odu.edu, kuhn@jlab.org}; Phone: \texttt{757 683 5804} \\

\textbf{Dr. Gail E. Dodge}, Professor, Old Dominion University, 4600 Elkhorn Ave, Room 306, Norfolk, VA 23529 \\
Email: \texttt{gdodge@odu.edu, gail@jlab.org}; Phone: \texttt{703 292 8958}\\

\textbf{Dr. Alexandre Deur}, Staff Scientist, Jefferson Lab, 12000 Jefferson Avenue, Newport News, VA 23606 \\
Email: \texttt{deurpam@jlab.org}; Phone: \texttt{757 584 7526}\\ 

\textbf{Dr. Marco Ripani}, Senior Scientist, INFN, Sezione di Genova, Via Dodecaneso 33, 16146, Genova, Italy \\
Email: \texttt{ripani@ge.infn.it, Marco.Ripani@ge.infn.it}; Phone: \texttt{+39-010-3536458}\\ 
\end{document}